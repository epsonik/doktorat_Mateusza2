\begin{figure}
\caption{Przykład wyników działania operacji average pooling i max
pooling z maską 2x2 i krokiem stride =2 na obrazie wynikowym o wy-
miarach 4x4}\label{tab:avg_pooling_vs_max}
\resizebox{\textwidth}{!}{
\begin{tabular}{ccc}  

\begin{tabular}[c]{@{}l@{}} \textbf{(a)} Wynikowa mapa aktywacji po\\ operacji splotowej, o rozmiarze 4x4. \end{tabular} & \begin{tabular}[c]{@{}l@{}} \textbf{(b)} Redukcja przez\\ maksimum\end{tabular} & \begin{tabular}[c]{@{}l@{}} \textbf{(c)} Redukcja przez\\ uśrednianie\end{tabular} \\
\begin{tabular}{|c|c|c|c|} 
\hline
1 & 3 & 2 & 4 \\ \hline
5 & 6 & 1 & 2 \\ \hline
7 & 2 & 3 & 6 \\ \hline
4 & 3 & 5 & 1 \\ \hline
\end{tabular} &
\begin{tabular}{|c|c|} 
\hline
6 & 4 \\\hline
7 & 6 \\\hline
\end{tabular} &
\begin{tabular}{|c|c|} 
\hline
3.75 & 2.25 \\\hline
4 & 3.75 \\ \hline
\end{tabular}

\end{tabular}}

\end{figure}